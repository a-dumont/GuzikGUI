\documentclass[openright,letterpaper,12pt]{book}
\usepackage[utf8]{inputenc}
\usepackage[french]{babel}
\usepackage{graphicx}
\usepackage{float}
\usepackage{pdfpages}
\usepackage[plainheadsepline,markcase=ignoreuppercase]{scrlayer-scrpage}
\usepackage{titlesec}
\usepackage{tocloft}
\usepackage{hyperref}
\usepackage{geometry}
\usepackage{listings}
\usepackage[dvipsnames]{xcolor}
\hypersetup{
		pdfauthor={Alexandre Dumont},
		pdftitle={MegaYoko},
		pdfsubject={Guide d'utilisateur},
		pdfkeywords={Manuel},
		%pdftex,
		colorlinks=true,
		breaklinks=true,
		urlcolor=RoyalBlue,
		linkcolor=RoyalBlue,
		citecolor=RoyalBlue,
		bookmarksopen=true,
		unicode=true
}
\usepackage{bookmark}
\usepackage{cmap} % Doit
\usepackage[T1]{fontenc}
\usepackage[hyperpageref]{backref}

\geometry{letterpaper, inner=1.5in, outer=1.0in, tmargin=1.5in, bmargin=1.0in, twoside=false}

\pagestyle{scrheadings}
%\renewcommand{\chaptermark}[1]{\markboth{{\thechapter. #1}}{}}
%\renewcommand{\sectionmark}[1]{aa}
\setlength{\headheight}{18pt}
\setlength{\footheight}{18pt}

\definecolor{codegreen}{rgb}{0,0.6,0}
\definecolor{codegray}{rgb}{0.5,0.5,0.5}
\definecolor{codepurple}{rgb}{0.58,0,0.82}
\definecolor{backcolour}{rgb}{1,1,1}

\lstdefinestyle{mystyle}{
	emph={get, output\_en, level},
    backgroundcolor=\color{backcolour},   
    commentstyle=\color{codegreen},
    keywordstyle=\color{codegreen},
    emphstyle=\color{codegreen},
    numberstyle=\tiny\color{codegray},
    stringstyle=\color{codepurple},
    basicstyle=\ttfamily\normalsize,
    breakatwhitespace=false,         
    breaklines=true,                 
    captionpos=b,                    
    keepspaces=true,                 
    numbers=left,                    
    numbersep=5pt,                  
    showspaces=false,                
    showstringspaces=false,
    showtabs=false,                  
    tabsize=2,
	aboveskip=20pt,
	belowskip=20pt,
}

\lstset{style=mystyle}

\begin{document}
\pagenumbering{gobble}
\includepdf[noautoscale=true, scale=1]{Figures/Pagetitre/pagetitre.pdf}
\clearpage\null\thispagestyle{empty}\clearpage
\thispagestyle{empty}
{\ } 

\vspace{3cm}
\noindent
{\fontsize{35.83pt}{40pt}\selectfont\bf Guzik-O-Scope\\} 
{\Large\bf Guide de l'utilisateur}

\vspace{10cm}\noindent
{\huge\bf Alexandre Dumont}\\
{\large ReuletLab}

\clearpage\null\thispagestyle{empty}

\renewcommand{\tableofcontents}
             {
                \clearpage
                \chapter*{\pdfbookmark[chapter]{\contentsname}{toc}
                          \contentsname}
                \csname @starttoc\endcsname{toc}
             }
\frontmatter
\chapter*{Garantie}
\addcontentsline{toc}{chapter}{Garantie}
Les instructions, le code et ce manuel sont fournis sans garanties et sans 
support et peuvent ne pas fonctionner.
\clearpage\null\thispagestyle{empty}

\tableofcontents
\clearpage\null\thispagestyle{empty}

\mainmatter

\clearpage
\chapter*{Installation}
\addcontentsline{toc}{chapter}{Installation}
Ce chapitre explique la procédure pour installer le code Python pour le 
Guzik-O-Scope. 
L'installation est facultative. 
Pour utiliser le code sans l'installer, il faut tout de même le télécharger et 
s'assurer que le code soit dans un répertoire accessible par l'instance de 
Python.

\section*{Obtenir les dépendances}
Pour pouvoir fonctionner, le Guzik-O-Scope à plusieurs dépendances, dont numpy 
et matplotlib, qui peuvent être facilement installée, si elles ne le sont pas 
déjà, et la librairie \verb+SignalProcessing+. 
Celle dernière est une librairie maison qui se trouve sur Github à l'adresse 
\href{https://github.com/a-dumont/SignalProcessing}
{https://github.com/a-dumont/SignalProcessing}. 
Pour installer cette libraire, il est nécessaire de compiler du code, soit à 
partir de cygwin sur Windows, où directement avec \verb+gcc+ sur Linux.

\subsection*{Windows}
Pour que le code puisse compiler correctement, certaines dépendances doivent 
êtres installées à partir de cygwin, notamment, \verb+CMake+, \verb+MinGW+, 
\verb+FFTW3+ et \verb+Pybind11+. 
Une fois ces dépendances installées via l'installateur de cygwin, il faut 
ouvrir
exécuter les commandes suivantes.
\begin{lstlisting}[language=Bash]
git clone https://github.com/a-dumont/SignalProcessing
cd SignalProcessing/
mkdir build && cd build
CXX=/usr/bin/x86_64-w64-mingw32-g++.exe cmake .. 
cmake --build . && cmake --install .
cd .. && /c/Anaconda3/python.exe setup.py install
\end{lstlisting}
Il est très important qu'a la dernière ligne, la commande \verb+python.exe+ 
soit celle du système et que son emplacement peut être différent de l'exemple 
fourni ici.
Puisqu'il faut installer les dépendances, les privilèges d'administrateur 
peuvent êtres requis. 
Si c'est le cas, le terminal cygwin doit être ouvert en tant qu'administrateur.

\subsection*{Linux}
Pour que le code puisse compiler correctement, certaines dépendances doivent 
êtres installées donc, \verb+CMake+, \verb+FFTW3+ et \verb+Pybind11+. 
Une fois ces dépendances installées, il faut 
exécuter les commandes suivantes.
\begin{lstlisting}[language=Bash]
git clone https://github.com/a-dumont/SignalProcessing
cd SignalProcessing/
mkdir build && cd build
cmake .. && cmake --build . && cmake --install .
cd .. && python setup.py install
\end{lstlisting}
Puisqu'il faut installer les dépendances, les privilèges d'administrateur 
peuvent êtres requis. 

\section*{Obtenir le code}
La façon la plus simple d'obtenir le code est d'utiliser \verb+git+ à partir 
d'un terminal, que ce soit sur Windows, macOS où Linux. 
Plus spécifiquement, il faut naviguer au répertoire choisi via le terminal et 
exécuter la commande suivante,

\begin{lstlisting}[language=Bash]
git clone https://github.com/a-dumont/GuzikGUI
\end{lstlisting}

Une façon alternative d'obtenir le code source est d'accéder au dépôt sur 
Github via un navigateur web et de télécharger le code dans une archive de 
type ZIP. 
Une fois cette archive décompresser dans le répertoire voulu, l'installation 
peut procéder normalement.

\section*{Installation}
Pour installer le code du Guzik-O-Scope comme un module Python, il faut 
naviguer dans le répertoire \verb+GuzikGUI+ à partir d'un terminal et lancer 
la commande,
\begin{lstlisting}[language=Bash]
python setup.py install
\end{lstlisting}
Il est possible que l'installation requiert les privilèges d'administrateur. 

\chapter*{Utilisation}
\addcontentsline{toc}{chapter}{Utilisation}
Le Guzik-O-Scope est fait pour fonctionner avec pyHegel et son implémentation 
du Guzik. 
Par contre, le code contient une classe permettant d'imiter le comportement du 
Guzik qui est instanciée automatiquement si le code ne parvient pas à 
charger l'implémentation de pyHegel. 
Ce cas est particulièrement pratique pour tester des modifications au code du 
Guzik-O-Scope sans avoir besoin du matériel physique.
\section*{Application seule}
Pour lancer le Guzik-O-Scope il suffit d'exécuter la commande suivante à partir 
d'un terminal.
\begin{lstlisting}[language=Bash]
python -m GuzikGUI
\end{lstlisting}
Il est possible d'écrire un script qui contient cette commande et de créer un 
raccourci pour celui-ci pour lancer l'application en un clic.

\section*{Application avec debug}
Pour une raison ou une autre, il peut être utile d'avoir accès aux variables 
qu'utilise le Guzik-O-Scope. 
Il est possible de lancer l'application via \verb+Ipython+ de manière 
interactive avec les commandes suivantes.
\begin{lstlisting}[language=Python]
from GuzikGUI import launch
application, window = launch()
\end{lstlisting}
L'objet \verb+window+ contient toute les variables intéressantes.
\clearpage\null\thispagestyle{empty}

\chapter*{Fonctionnalités}
\addcontentsline{toc}{chapter}{Fonctionnalités}


\end{document}
